\section{Tabele}
\subsection{Firmy}
Przechowuje informacje o firmach: numer firmy, nazwa firmy, (opcjonalny) NIP.
\begin{minted}[frame=lines, linenos]{sql}
CREATE TABLE CompanyCustomers (
    CustomerID int  NOT NULL,
    CompanyName nvarchar(64)  NOT NULL,
    NIP varchar(16)  NULL,
    CONSTRAINT CompanyCustomers_pk PRIMARY KEY  (CustomerID)
);
\end{minted}
\section{Procedury}
\subsection{UpdateConstants(...)}
Aktualizuje podane stałe (niezmieniając pozostałych)
\begin{minted}[frame=lines, linenos]{sql}
CREATE PROCEDURE UpdateConstants(
    @Z1 INT = NULL,
    @K1 INT = NULL,
    @R1 INT = NULL,
    @K2 INT = NULL,
    @R2 INT = NULL,
    @D1 INT = NULL,
    @WZ INT = NULL,
    @WK INT = NULL
) AS BEGIN
    DECLARE @PREV_Z1 INT
    DECLARE @PREV_K1 INT
    DECLARE @PREV_R1 INT
    DECLARE @PREV_K2 INT
    DECLARE @PREV_R2 INT
    DECLARE @PREV_D1 INT
    DECLARE @PREV_WZ INT
    DECLARE @PREV_WK INT

    SELECT 
        @PREV_Z1 = Z1,
        @PREV_K1 = K1,
        @PREV_R1 = R1,
        @PREV_K2 = K2,
        @PREV_R2 = R2,
        @PREV_D1 = D1,
        @PREV_WZ = WZ,
        @PREV_WK = WK
    FROM Constants

    INSERT INTO Constants(Date, Z1, K1, R1, K2, R2, D1, WZ, WK)
    VALUES (
        GETDATE(),
        ISNULL(@Z1, @PREV_Z1),
        ISNULL(@K1, @PREV_K1),
        ISNULL(@R1, @PREV_R1),
        ISNULL(@K2, @PREV_K2),
        ISNULL(@R2, @PREV_R2),
        ISNULL(@D1, @PREV_D1),
        ISNULL(@WZ, @PREV_WZ),
        ISNULL(@WK, @PREV_WK)
    )
END
GO
\end{minted}
\subsection{AddCompanyCustomer}
Dodaje firmę jako klienta.
\begin{minted}[frame=lines, linenos]{sql}
CREATE PROCEDURE AddCompanyCustomer(
    @Email nvarchar(64),
    @Phone nvarchar(16),
    @Address nvarchar(64),
    @City nvarchar(64),
    @PostalCode varchar(16),
    @Country nvarchar(64),
    @CompanyName nvarchar(64),
    @NIP varchar(16)
)
AS BEGIN
    INSERT INTO Customers (Email, Phone, Address, City, PostalCode, Country)
    VALUES (@Email, @Phone, @Address, @City, @PostalCode, @Country)
    INSERT INTO CompanyCustomers (CustomerID, CompanyName, NIP)
    VALUES (@@IDENTITY, @CompanyName, @NIP)
END
GO
\end{minted}
\subsection{AddPrivateCustomer}
Dodaje osobę prywatną jako klienta.
\begin{minted}[frame=lines, linenos]{sql}
CREATE PROCEDURE AddPrivateCustomer(
    @Email nvarchar(64),
    @Phone nvarchar(16),
    @Address nvarchar(64),
    @City nvarchar(64),
    @PostalCode varchar(16),
    @Country nvarchar(64),
    @FirstName nvarchar(64),
    @LastName nvarchar(64)
)
AS BEGIN
    INSERT INTO Customers (Email, Phone, Address, City, PostalCode, Country)
    VALUES (@Email, @Phone, @Address, @City, @PostalCode, @Country)
    INSERT INTO PrivateCustomers (CustomerID, FirstName, LastName) 
    VALUES (@@IDENTITY, @FirstName, @LastName)
END
GO
\end{minted}
\subsection{CreateOrderInvoice(OrderID)}
Generuje fakturę w tabeli Invoices dla danego zamówienia.
\begin{minted}[frame=lines, linenos]{sql}
CREATE PROCEDURE CreateOrderInvoice(@OrderID int)
AS BEGIN
    IF (SELECT InvoiceID FROM Orders WHERE OrderID = @OrderID) IS NOT NULL
    BEGIN
        ;THROW 5200, 'Order already has an invoice', 1
        RETURN
    END
    IF (SELECT Paid FROM Orders WHERE OrderID = @OrderID) = 0
    BEGIN
        ;THROW 5200, 'Order hasnt been paid', 1
        RETURN
    END

    INSERT INTO Invoices(
        Date, TotalAmount, FirstName, LastName, CompanyName, Email, Phone, Address, City, PostalCode, Country
    )
    SELECT GETDATE(), dbo.TotalOrderAmount(@OrderID), FirstName, LastName, CompanyName, 
                Email, Phone, Address, City, PostalCode, Country 
    FROM Orders
        JOIN Customers ON Customers.CustomerID = Orders.OrderID
        LEFT JOIN CompanyCustomers ON CompanyCustomers.CustomerID = Customers.CustomerID
        LEFT JOIN PrivateCustomers ON PrivateCustomers.CustomerID = Customers.CustomerID
    WHERE Orders.OrderID = @OrderID;

    UPDATE Orders SET InvoiceID = @@IDENTITY
    WHERE OrderID = @OrderID
END
GO
\end{minted}
\subsection{CreateMonthlyInvoice(CustomerID, Month, Year)}
Generuje fakturę dla danego klienta, dla danego miesiąca.
\begin{minted}[frame=lines, linenos]{sql}
CREATE PROCEDURE CreateMonthlyInvoice(@CustomerID Int, @Month int, @Year int)
AS BEGIN
    -- Last day of the month is in the past
    IF DATEFROMPARTS(@Year, @Month, DAY(EOMONTH(DATEFROMPARTS(@Year, @Month, 1)))) >= GETDATE()
    BEGIN
        ;THROW 5200, 'The month hasnt passed yet', 1
        RETURN
    END

    INSERT INTO Invoices(
        Date, TotalAmount, FirstName, LastName, CompanyName, Email, Phone, Address, City, PostalCode, Country
    )
    SELECT GETDATE(), SUM(dbo.TotalOrderAmount(Orders.OrderID)), MAX(FirstName), MAX(LastName), 
            MAX(CompanyName), MAX(Email), MAX(Phone), MAX(Address), MAX(City), MAX(PostalCode), MAX(Country) 
    FROM Customers
        LEFT JOIN Orders ON Orders.CustomerID = Customers.CustomerID AND Orders.InvoiceID IS NULL
                            AND MONTH(Orders.CompletionDate) = @Month AND YEAR(Orders.CompletionDate) = @Year
        LEFT JOIN CompanyCustomers ON CompanyCustomers.CustomerID = Customers.CustomerID
        LEFT JOIN PrivateCustomers ON PrivateCustomers.CustomerID = Customers.CustomerID
    WHERE Customers.CustomerID = @CustomerID
    GROUP BY Customers.CustomerID

    UPDATE Orders SET InvoiceID = @@IDENTITY
    WHERE Orders.CustomerID = @CustomerID AND Orders.InvoiceID IS NULL
            AND MONTH(Orders.CompletionDate) = @Month AND YEAR(Orders.CompletionDate) = @Year
END
GO
\end{minted}
\subsection{NewMenuInProgress(StartDate, EndData, MenuID OUTPUT)}
Tworzy nowe nieaktywne menu.
\begin{minted}[frame=lines, linenos]{sql}
CREATE PROCEDURE NewMenuInProgress(@StartDate datetime, @EndDate datetime, @MenuID int OUTPUT)
AS BEGIN
    INSERT INTO Menu(StartDate, EndDate, Active)
    VALUES(@StartDate, @EndDate, 0)
    
    SET @MenuID = @@IDENTITY
END
GO
\end{minted}
\subsection{ChangeMenuDates(MenuID, StartDate, EndDate)}
Zmienia datę niaktywnego menu.
\begin{minted}[frame=lines, linenos]{sql}
CREATE PROCEDURE ChangeMenuDates(@MenuID int, @StartDate datetime = NULL, @EndDate datetime = NULL)
AS BEGIN
    IF (SELECT Active FROM Menu WHERE MenuID = @MenuID) = 1
    BEGIN
        ;THROW 52000, 'Menu is active', 1
        RETURN
    END 

    DECLARE @PrevStartDate datetime
    DECLARE @PrevEndDate datetime

    SELECT @PrevStartDate = StartDate, @PrevEndDate = EndDate 
    FROM Menu WHERE MenuID = @MenuID

    UPDATE Menu
    SET StartDate = ISNULL(@StartDate, @PrevStartDate),
        EndDate = ISNULL(@EndDate, @PrevEndDate)
    WHERE MenuID = @MenuID
END
GO
\end{minted}
\subsection{SetMenuItem(MenuID, MealID, Price)}
Dodaje posiłek do nieaktywnego menu.
\begin{minted}[frame=lines, linenos]{sql}
CREATE PROCEDURE SetMenuItem(@MenuID int, @MealID int, @Price money = NULL)
AS BEGIN
    IF (SELECT Active FROM Menu WHERE MenuID = @MenuID) = 1
    BEGIN
        ;THROW 52000, 'Menu is active', 1
        RETURN
    END 

    IF (SELECT Active FROM Meals WHERE MealID = @MealID) = 0
    BEGIN
        ;THROW 52000, 'Meal is not active', 1
        RETURN
    END 

    DECLARE @DefaultPrice money = (SELECT DefaultPrice FROM Meals WHERE MealID = @MealID)

    INSERT INTO MenuItems(MenuID, MealID, Price)
    VALUES (@MenuID, @MealID, ISNULL(@Price, @DefaultPrice))
END
GO
\end{minted}
\subsection{RemoveMenuItem(MenuID, MealID)}
Usuwa posiłem z nieaktynwgo menu.
\begin{minted}[frame=lines, linenos]{sql}
CREATE PROCEDURE RemoveMenuItem(@MenuID int, @MealID int)
AS BEGIN
    IF (SELECT Active FROM Menu WHERE MenuID = @MenuID) = 1
    BEGIN
        ;THROW 52000, 'Menu is active', 1
        RETURN
    END

    DELETE MenuItems
    WHERE MenuID = @MenuID AND MealID = @MealID
END
GO
\end{minted}
\subsection{ActivateMenu(MenuID)}
Próbuje aktywować menu biorąc pod uwagę niepowtarzanie się posiłków i nienachodzenie dat.
\begin{minted}[frame=lines, linenos]{sql}
CREATE PROCEDURE ActivateMenu(@MenuID int)
AS BEGIN
    -- Check if not active
    IF (SELECT Active FROM Menu WHERE MenuID = @MenuID) = 1
    BEGIN
        ;THROW 52000, 'Menu is active', 1
        RETURN
    END

    -- Check if date do not overlap/ there is a gap
    DECLARE @StartDate datetime = (SELECT StartDate FROM Menu WHERE MenuID = @MenuID)
    DECLARE @LastMenuDate datetime = (SELECT MAX(EndDate) FROM Menu WHERE Active = 1)
    
    if DATEDIFF(day, @LastMenuDate, @StartDate) <= 0
    BEGIN
        ;THROW 52000, 'Overlapping dates', 1
        RETURN
    END

    -- Check if there menu items are legal
    DECLARE @Count int
    DECLARE @NotChangedCount int

    SELECT @Count = Count(MealID)
    FROM Menu
    JOIN MenuItems ON MenuItems.MenuID = Menu.MenuID
    WHERE Menu.MenuID = @MenuID

    SELECT @NotChangedCount = Count(MealID)
    FROM Menu 
    JOIN MenuItems ON MenuItems.MenuID = Menu.MenuID
    WHERE Menu.MenuID = @MenuID AND MenuItems.MealID IN (
        SELECT MI2.MealID
        FROM MenuItems AS MI2
        JOIN Menu AS M2 ON M2.MenuID = MI2.MenuID
        WHERE M2.Active = 1 AND DATEDIFF(day, M2.EndDate, Menu.StartDate) < 14
    )

    IF (@NotChangedCount * 2) > @Count
    BEGIN
        ;THROW 52000, 'Menu is not legal', 1
        RETURN
    END

    -- Everything is correct
    UPDATE Menu SET Active = 1
    WHERE MenuID = @MenuID
END
GO
\end{minted}
\section{Widoki}
\subsection{MenusInProgress}
Pokazuje nie aktywne menu.
\begin{minted}[frame=lines, linenos]{sql}
CREATE VIEW MenusInProgress AS
SELECT MenuID FROM Menu WHERE Active = 0
GO
\end{minted}
\section{Funkcje}
\subsection{TotalOrderAmount(OrderID)}
Zwraca całkowitą cenę zamówienia biorąc pod uwagę rabaty.
\begin{minted}[frame=lines, linenos]{sql}
CREATE FUNCTION TotalOrderAmount(@OrderID int) RETURNS money
BEGIN
    RETURN (
        SELECT SUM(OD.Number * MI.Price) * (1-SUM(Discounts.Discount)) FROM Orders
        JOIN OrderDetails AS OD ON OD.OrderID = Orders.OrderID
        JOIN MenuItems AS MI ON MI.MenuID = OD.MenuID AND MI.MealID = OD.MealID
        JOIN OrderDiscounts AS Discounts ON Discounts.OrderID = Orders.OrderID
        WHERE Orders.OrderID = @OrderID
        GROUP BY Orders.OrderID
    )
END
GO
\end{minted}
